%Arquivo contendo o capítulo Introdução
\chapter{Introdução} \label{cap:introducao}
Na área da tecnologia da informação, os dados de logs são o registro definitivo sobre o que está acontecendo em um sistema de informação. Os logs gerados geralmente são dados não estruturados e que não possuem um padrão definido entre eles, o que torna a extração de informações destes logs uma tarefa complexa.. \par 

Com o aumento do uso de sistemas de informação no mundo corporativo, as empresas possuem um maior volume de dados de logs, consequentemente um maior volume de informações para auxiliar a empresa em uma suposta resolução de problemas, e também em tomadas de decisões de negócio. Porém, a maioria das empresas acabam negligenciando o potêncial das informações que podem ser extraídas destes dados de logs. Alguns dos motivos disso são:

\begin{itemize}
\item Alta complexidade em análizar o grande volume de dados gerados pelos logs.
\item Alta complexidade em normalizar e categorizar logs de diferentes formatos e fontes.
\item Alta complexidade em relacionar os logs que são gerados de diferentes fontes.
\item Dificuldade de entender os benefícios que podem ser adquiridos com o investimento.
\end{itemize}

Geralmente, as empresas que investem na extração de dados de logs, usam um ou mais sistemas de informação que tem como sua principal funcionalidade a extração e monitoramento de informações dos dados de logs, independente da fonte que o mesmo foi gerado. Com isso a empresa consegue ter diversos benefícios, dentre eles: 

\begin{itemize}
\item Reduzir o tempo de inatividade inesperado de um sistema.
\item Facilitar a análise e solução de problemas, e identificar a causa raíz destes problemas.
\item Evita desgastes entre as equipes de TI no momento em que as equipes trabalham juntas para solucionar o problema detectado a partir dos dados de logs, ao inves de as equipes acusarem umas as outras pelo problema.
\item Evitar a perda de tempo para detectar erros de forma manual em uma grande massa de dados de logs, fazendo com que o sistema responsável faça isso automaticamente pelo responsável por análizar os logs.
\item Visibilidade geral da saúde dos sistemas.
\end{itemize}

Este trabalho tem como objetivo desenvolver uma solução, que irá proporcionar a centralização de dados de logs, de diferentes sistemas, em um local unico, para que esses dados possam ser análizados e apresentados para os usuários para que se tenha os beneficios citados anteriormente. Para realizar isso, a solução a ser desenvolvida, irá receber estes dados de logs gerados por um ou mais sistemas de informação, e centralizar essa massa de dados em uma base de dados unica. Essa centralização dos dados, será feita utilizando serviço que será desenvolvido e que consegue fazer uma análise nos dados de logs, que geralmente são arquivos textos, baseado em uma configuração pré-definida pelo usuário da ferramenta, onde o usuário especifica o formato dos logs gerados por cada sistema. \par

Alem da centralização dos dados a solução terá uma aplicação Web, onde o usuário poderá:
\begin{itemize}
\item Visualizar os logs, sendo possível utilizar filtros diversos.
\item Criar e salvar dashboards que poderão conter um ou mais gráficos, onde o usuário poderá escolher a relação entre os tipos de dados e campos.
\item Carregar dashboards salvos e monitora-los em tempo real.
\end{itemize}

Com a solução proposta desenvolvida, teremos uma forma de unificar os dados de logs gerados por diferentes sistemas, e uma interface onde seja possível extrair as informações destes dados de uma maneira mais rápida e eficiente comparando a uma extração de informações manual dos dados.

\section{Motivação}
Paragrafo1 \par

Paragrafo2

\section{Objetivo}
Este trabalho possui os seguintes objetivos:

\subsection{Objetivo Geral}
Desenvolver um \textit{software} com a pretensão de motivar o usuário a persistir praticando a atividade de ciclismo. 

\subsection{Objetivos Específicos}
\begin{itemize}
\item Identificar quais dados devem ser analisados e como computá-los para mensurar o desempenho do ciclista;
\item Identificar e selecionar quais \textit{frameworks} e linguagens de programação adequados para o desenvolvimento do \textit{software};
\item Desenvolver uma interface homem-computador responsiva e multiplataforma;
\item Desenvolver um mecanismo dentro da aplicação para propor desafios ao usuário e recompensá-lo pela participação nestes;
\item Permitir ao usuário compartilhar seus dados contidos no \textit{software} em redes sociais;
\item Analisar e aplicar técnicas de gamificação para o desenvolvimento do \textit{software}.
\end{itemize}

